\documentclass[12pt]{article}

\usepackage{fullpage}
\usepackage{graphicx}
\usepackage{hyperref}
\usepackage{listings}
\usepackage{multirow}
\usepackage{pdflscape}

\begin{document}

\setcounter{secnumdepth}{0}

\title{Smart Police Body Camera}
%\subtitle{EECS 473}
\author{
    Yun Chan Han \and
    Jiqing Jiang \and
    Alex LaBerge \and
    Jacob Perrin \and
    Alec Ten Harmsel
}
\date{}
\maketitle

\newpage

% TODO argumentation piece - benefits, camera, wifi vs. cell

\section{Summary}
Every year around 500 people are killed by law enforcement personnel. In 2015,
especially, people are wondering how many of these deaths were preventable. One
classic way to ensure that all protocols were being followed has been a police
body camera. However, these are often expensive (\$800 - \$1200 \cite{cam})
each and have serious privacy concerns for the police officers. If the body
cameras are always on, you might record private moments that an officer does
not want his superiors seeing. However, if we give the officer a way to turn it
on and off and police abuse may start occurring. The proposed device would be
an upgrade to the current system: a way to measure and record police weapon
usage as well as a way to keep track of the officers and their behavior when
weapons are pulled without violating their privacy when monitoring may be
unnecessary.

This product will be helpful because it will hopefully be cheaper than the
current alternative (regular body cameras) while also containing more features
that will be beneficial to the suspect being stopped/arrested as well as
beneficial to the officers' safety and privacy. This product will see use
because it hits a middle ground that city officials and law enforcement
officials can agree upon. It will be cost-effective, while also being robust
and hopefully life-saving.

\section{Description}

This design needs to be cheaper to produce than current body-cam systems as
well as being accurate in how and why it records data (in a manner which keeps
officers' private information private). We do not want to capture the call they
have in the squad car with their wives, but we do want to capture the stop
where a suspect gets out of their car and attempts to attack the officer.
Police will be more likely to use it for these features. In addition, storing
footage from similar body cameras that are always on represents a significant
challenge for police departments\cite{store1,store2}.

Some of the solutions that are already on the market include many more features
than are feasible for the scope of this project. Wolfcom has a product called
the 3rd Eye\cite{third_eye} which has a snapshot button, a screen for seeing
what the camera sees, and HDMI support. However, it comes at a steep price
point of \$475.  However, it does not include streaming and many of the
features seem to be from the perspective of a swiss-army-knife of cameras
rather than a device to stop police brutality.  They have complete control over
the camera and therefore could abuse the technology. Our solution would have
similar resolution, while leaving the control up to the environment and would
give instant feedback to the police station about the officer's behaviour.

Both of our proposed solutions require a body-mounted camera and a pack located
on the user’s hip that interfaces with both the camera and a possible custom
holster. From there, the designs differ on how they process and store the
incoming data.
        
In one of the proposed designs, the camera will take a picture once the gun has
been brought up to be fired, as well as taking pictures on each trigger pull.
These pictures will be sent back to the pack on the user’s hip and then
wirelessly streamed over to the computer within the squad car. 

In the other proposed design, the camera will start recording video once the
gun has been removed from the holster and will send video to be stored locally
on the officer’s hip pack that will be streamed over when a reliable connection
can be established. In both designs, we will use a GPS device for tracking the
officer as well as synchronizing the video or images with the current time,
which is included in the GPS protocol. The way we will tell if the gun is being
brought up to be fired will be with a gyroscope attached to the weapon. An
alternative to this option would be a holster mechanism which will tell if a
gun has been pulled by checking if an RFID tag can be read. This RFID tag will
have a range such that it only reads the tag if the gun when it is in the
holster. If it cannot be read, the gun is not holstered. The video feed would
then begin to record.

The tradeoffs are fairly simple. For the design that takes pictures, storage
would be much easier as well as wireless transmission of the pictures. The
design would probably not be as power hungry either, as streaming data will be
a huge consumer of battery. However, if the picture is blurry or we got a
picture at a point where the police officer is not facing the suspect, we are
out of luck. We also would not likely have synchronized sound. The video
design, on the other hand, will be much harder to implement due to the
streaming of video. We will probably need a Linux-based (or other higher-level
“RTOS”) processor to interface with Wi-Fi as well as to make streaming easier.
This would be a pain to wake up every time we needed to start recording on the
camera. Due to the fact that we would probably always need to keep the camera
on but not recording, power consumption becomes one of the bigger problems.
Between constantly keeping the camera on and the amount of power required to
run an RFID reader and microphone algorithm, we could run into many issues
trying to power our device. Some of the difficulties of implementing both
designs will be discussed next.

\subsection{Snapshot Solution}

The implementation difficulties related to this solution would be the
mechanical portions of determining a trigger pull. The rest of this solution
would be close to trivial, as sending a picture is a lot simpler than streaming
data back to a base station. Another mechanical issue would be how we would
attach the gyroscope onto the weapon without being obstructive in any way. The
gyroscope would have to wirelessly transmit the data over and that could be
slow, perhaps missing an important shot between wireless and camera delay. This
wireless solution would also add to the bulk of the gun-based solution.

\subsection{Video Feed Solution}

Any device that streams video consumes a large portion of processing and power.
This would be a challenge for a device that needs to be small enough to sit on
someone’s belt. Also, we would have to interface with a video encoder/decoder
in order to stream efficiently, and increasing the number of devices can always
lead to problems. This would likely require a separate processor dedicated
solely for this processing, as it would have to be on only when the video is
being recorded and sent in order to conserve power. This means we would have to
interface between processors in addition to the other issues.

\subsection{Camera Mount Solution}

Most of mounting in the industry use clips to attach a camera on a body. Among
different clipping mounts, positions where they are clipped to vary. We have
sorted out to three possible positions where a camera can be clipped to:
sunglass, chest, and shoulder. After comparing pros and cons of each, we have
decided to use the clip mounted to shoulder as our mount type as it has
moderate accuracy and stability without much obstruction. The comparisons of
these options are summarized in the table below.

\begin{table}[h!]
    \centering
    \caption{Comparison of different types of camera mounts}
    \begin{tabular}{|l|l|c|}
        \hline
        \textbf{Mount Type} & \textbf{Pros and Cons} & \textbf{Mounted Image}\\
        \hline
        & & \\
        Sunglasses Clip & Pros: & \multirow{8}{*}{\includegraphics[width=0.3\textwidth]{glasses_mount}}\\
                        & - Camera direction corresponds to LOS & \\
                        & - Most accurate in camera's view & \\
                        & - Stable & \\
                        & & \\
                        & Cons: & \\
                        & - Has to wear glasses & \\
                        & - Can be disturbing & \\
        & & \\
        \hline
        & & \\
        Chest Clip & Pros: & \multirow{8}{*}{\includegraphics[width=0.3\textwidth]{chest_mount}}\\
                   & - Not disturbing & \\
                   & & \\
                   & Cons:  & \\
                   & - View can be blocked by officer' arms & \\
                   & - Can be unstable upon active movement  & \\
                   & & \\
                   & & \\
                   & & \\
        \hline
        & & \\
        Shoulder Clip & Pros: & \multirow{7}{*}{\includegraphics[width=0.3\textwidth]{shoulder_mount}}\\
                      & - Moderately accurate in camera's view & \\
                      & - View is unlikely to be blocked & \\
                      & & \\
                      & Con: & \\
                      & - Can be unstable upon active movement  & \\
                      & & \\
        \hline
    \end{tabular}
\end{table}

\begin{figure}[h!]
    \centering
    \includegraphics[width=0.9\textwidth]{installation}
    \caption{Overview of Installation}
\end{figure}

\begin{figure}[h!]
    \centering
    \includegraphics[width=0.9\textwidth]{state_diagram}
    \caption{Camera Record State Diagram}
\end{figure}

\begin{table}[h!]
    \centering
    \caption{Summary of Power Consumption}
    \begin{tabular}{lr}
        \textbf{Device} & \textbf{Power Consumption (W)}\\
        Video MCU & 0.3\\
        Camera & 3.0\\
        GPS & 0.3\\
        Microphone & 1.0\\
        RFID Reader & 0.1\\
        Main MCU & 0.1\\
        Total & 4.8\\
        Total less Video Subsystem & 1.5\\
    \end{tabular}
\end{table}

In an 8 hour police shift, our device can record up to one half hour of video,
requiring approximately 13.65 Wh of energy. A battery with ~20 Wh has been
chosen for overhead and to deal with wear in production use.

\section{Implementation Issues}

The device revolves around two main parts, the belt enclosure and the camera.
The holster sensor will be an RFID reader which, when it cannot read the RFID
tag on the weapon, will send a signal to the main processor. This processor
will be located on the belt in an enclosure along with any other processors we
have. The main processor will then send a signal to begin gathering data
through a video camera that we have attached to the officer. 

The video will be recorded onto an SD card and streamed back when a wireless
signal is available. We are planning on setting up a Raspberry Pi or similar
processor as a wireless access point for Wi-Fi, so we could attach a Wi-Fi
module onto the wearable and stream the video data back over that connection.
This streaming might require a dedicated processor, which would also go in the
hip enclosure. One processor that might be able to work is the TIVA, made by
TI. MAAV, the Michigan Autonomous Aerial Vehicle, uses TIVAs and USB webcams to
stream HD video, so we figured it is definitely doable on this platform. The
video would be encoded by this chip and then sent over a Wifi radio with a UART
interface.

We will also have a microphone interfaced to the main processor, which will
also send a record signal if it detects loud noises. This is to potentially
capture any violent scenarios that do not include a weapon being drawn. This
microphone will probably not save any data, but will instead be used solely for
the purpose of detecting a decibel altitude above a certain threshold as a
trigger to begin capturing video.

We will also have a Real Time Clock (RTC) on the device, which will be used to
timestamp the videos. To sync the RTC, we will be using a GPS device to get the
time from that protocol. This has an added bonus of also giving us the GPS
coordinates of the police officer. The reason we cannot use solely GPS is that
signal may not be available indoors. This timestamp data will be added as
metadata to the video before it is sent off to be streamed. 

For the camera, we have a few potential devices. We could use a premade camera
that saves data onto an SD card and interface with that, or we could use
something like an IP camera that already has network-level streaming available.
Both of these options are possibilities, but having an IP camera would limit
how low-level we can stay with the camera processing and will probably force us
into using something like embedded Linux (or worse). The foreseen issue with
embedded Linux is the lengthy startup time and the tradeoff between the power
of being up at all time or latency of boot. This may not make Linux an ideal
solution. A premade camera like a GoPro would be expensive and larger than what
we need. It would likely use more power than a smaller dedicated camera module
that we interface with manually. This would give use even less ability to
customize to our suit our needs than the IP camera. In addition, a
manufacturable product may run into copyright issues. Using an embedded camera
device would be difficult to interface with: in getting the images initially,
storing them in a meaningful way, and then streaming them. The device that we
settled on after all of our research is an HD webcam with a USB interface that
can be sent to the video encoding processor we have selected.

% TODO move this
\begin{table}[h!]
    \centering
    \caption{Comparison of finished products}
    \begin{tabular}{llll}
        \textbf{Name} & Team Edgy & PatrolEyes SC-DVAI & Wolfcom Vision\\
        \textbf{Price} & \textasciitilde \$300 & \$449 & \$249\\
        \textbf{Video Capacity} & 32 GB & 32 GB & 32 GB\\
        \textbf{Video Resolution} & 1080p & 1080p & 1080p\\
        \textbf{Video Format} & MOV & AVI & MOV, MPEG-4\\
        \textbf{Photo Resolution} & N/A & 16M, 12M, 5M, 3M & 16M, 12M, 8M, 5M, 3M\\
        \textbf{Standby Time} & 10 hours & 35 hours & $>$ 5 days\\
        \textbf{Remote Control} & No & Yes & No\\
        \textbf{Field of View} & $74^o$ & $120^o$ & $120^o$\\
        \textbf{Mass} & ? & 151g & 63g\\
        \textbf{Infrared Camera} & No & Yes & No\\
        \textbf{Dimensions} & ? & DVR: 79 x 51 x 22 mm & 74 x 38 x 15 mm\\
                            & & Camera: 55 x 48 x 38 mm & \\
    \end{tabular}
    \label{tab:fin_comp}
\end{table}

We have debated a few different methods for detecting the status of the gun. We
originally discussed using an ultrasonic sensor in the holster to detect if
something is blocking the sensor, but that would be easy to trick by stuffing
something in the holster in order to turn the video feed off, which is the kind
of tampering we are trying to stop. We discussed RFID, which would involve a
tag being placed on the gun and a reader being mounted to the holster. This
method would be nearly impossible to fool and thus gives greater security. The
limitation of this method could be the complexity and power-draw, but
ultimately this seems to be the most elegant and least intrusive. In addition,
finding an RFID reader that consumes a small amount of power but also gives us
the range to detect whether the gun has been removed from the holster might be
a challenge. We have found a few devices that have reasonable power draw and
will satisfy our range requirement of a few centimeters, but the signal having
to go through the holster might affect some of the numbers. The device that we
settled on for detecting the gun status is an RFID reader. This gives us a
definitive way of determining the position of the gun that is tamper-proof.

For the problem of mounting the camera, we have decided on mounting it on the
officer's shoulder. From the compare and contrast section we did previously
about mounting locations, we feel that the pros strongly outweigh the cons. For
testing purposes, we do not need to perform strenuous activity that might cause
the shoulder mount to not capture useful video. We also feel like requiring the
officer to wear glasses, especially in a nighttime scenario, would cause
problems and we felt like the chest mount would often be obscured by the way
the officer should shoot his weapon.

\begin{figure}[h!]
    \centering
    \includegraphics[width=0.9\textwidth]{overview}
    \caption{High Level Overview}
\end{figure}

\section{Mechanical Design}

The mechanics of this design revolve around one enclosure that contains the PCB
that houses the MSP430, GPS, RTC, RFID reader, an exposed switch, and a
microphone. The RFID reader will have an antenna coming out of the enclosure
that will be mounted on the holster. This enclosure will also contain a TIVA
with an attached WiFi radio, which will also have an externally mounted
antenna. The camera will be mounted on the shoulder of the officer and will be
connected to the enclosure via a USB cable, as the camera solution we have is
USB compatible. The camera will also have an enclosure that allows it to
comfortably be mounted to the officer’s shoulder. All connections and the
protocols they use can be found in the diagram above.

Another mechanical design feature is the size limitations we will run into.
Officers already carry weapons as well as radios on their belt, so anything we
add to the belt will not be too much for the officer as long as it is not big
enough to be physically intrusive. However, sticking to just a camera mounted
on the shoulder is key, because something heavy on a shoulder could knock his
aim off and potentially put the officer in a dangerous situation. 

\section{Way Forward}

In order to implement our design, we will adhere as closely as possible to the
Gantt chart in Figure \ref{fig:gantt}. The things that are most likely to go
wrong will be developing a streaming driver for video as well as developing the
RFID controller. Our schedule leaves approximately a week to two weeks for
``final manufacturing'', which will be implementing our enclosures and possibly
researching manufacturability and production cost. Manufacturing an enclosure
for our prototype will be fairly simple, because Alec has a lot of experience
working with the 3D printing process. In those few weeks, testing could occur
on any unfinished section of the project. 

\begin{landscape}
\begin{figure}[h!]
    \centering
    \includegraphics[width=1.0\textwidth]{gantt}
    \caption{Gantt}
    \label{fig:gantt}
\end{figure}
\end{landscape}

\section{Milestones}

\subsection{Milestone 1}

Milestone 1 will be on October 20, where we should have large chunks of many of
the parts of our project done. By this time, we want our MSP430, either the
RFID reader model we hope to get donated from TI or just a regular one that can
be found in the MESH lab. This processor will generate a record signal based on
volume levels detected by a microphone. The record signal will also be set by a
switch that we interface to, and the RFID reader will hopefully be close to
done.We should have the camera recording and sending data back to the TIVA, as
well as a demo for sending data across a Wifi connection. The Wifi demo might
be as simple as sending one packet with a string in it, but we will have the
connection and that will give us a good platform for sending over the encoded
video data soon after. The MSP430 will also be reading in GPS and RTC data at a
reasonable rate. This data can be checked via the internet by typing in the
coordinates we receive from the device and checking the time with a clock. 

\subsection{Milestone 2}

Milestone 2 will be on November 4, where we should have the breadboard
prototype completely finished, with video being recorded and streamed over to
the Pi whenever there is a loud noise or the gun is removed from the holster,
as well as if the switch is toggled. All of this will be on a breadboard
prototype, with the PCB having been designed and shipped, but not received.
With the PCB hopefully received at this point, we can start testing and
designing enclosures.

\section{Budget and Bill of Materials}

\begin{table}[h!]
    \centering
    \caption{Budget and Bill of Materials}
    \begin{tabular}{|lrrllr|}
        Product & Cost & Qty & Manufacturer & URL/description & Shipping\\
        Camera & \$69.00 & 1 & Logitech & \url{http://www.newegg.com/Product/Product.aspx?Item=9SIA4A02Z35722} & \$0.00\\
        GPS & \$39.95 & 1 & USGlobalSat & \url{https://www.sparkfun.com/products/12751} & \$2.50\\
        RTC & \$3.47 & 1 & Epson & \url{http://www.digikey.com/product-detail/en/RX8900CE:UA3/SER4053CT-ND/5175145} & \$3.00\\
        Microphone & \$7.95 & 1 & Challenge Electronics & \url{https://www.sparkfun.com/products/9964} & \$2.50\\
        Wifi Radio & \$6.95 & 1 & Nurdspace & \url{https://www.sparkfun.com/products/13678} & \$3.00\\
        RFID Reader Eval Board & \$0.00 & 1 & TI (hopefully donated) & \url{http://www.digikey.com/product-search/en?keywords=eZ430-TMS37157} & \$0.00\\
        Video Encoding Processor & \$15.61 & 1 & TI & \url{http://www.ti.com/product/TM4C1290NCPDT/technicaldocuments} & \$3.04\\
        PCB fabrication & \$33 & 2 & AC & \url{http://www.4pcb.com/pcb-student-discount.html} & \$10.00\\
        PCB Parts & \$100 & 1 & Digikey & Various & \$20.00\\
        Battery & \$37 & 1 & Venom & \url{http://www.amazon.com/Venom-3200mAh-Battery-Universal-System/dp/B00177VKS6/ref=sr_1_25?ie=UTF8&qid=1444205971&sr=8-25&keywords=2s+lipo+battery} & \$0.00\\
        Total & \$312.93 & & & 44.04\\
    \end{tabular}
\end{table}

\section{Demoables for Design Expo}

Ideally, we will have a base station (a laptop or a Pi) to simulate the squad
cars laptop. The Pi will be a Wifi access point that our wearable will connect
to. From that connection we get a video stream whenever the gun is pulled from
the holster or whenever someone screams or yells near the microphone. We could
also show the hip pack that is tamper-proof. When the gun is not holstered, it
will only show GPS data from the officer’s pack. This could involve possibly
creating a halfway-decent UI, if we have time.

\section{Conclusion}

Our project proposes an optimal scheme to satisfy questions of unethical
conduct by police officers. Even though police officers maintain the right to
apprehend suspects by force, many suspect officers of abuse of their power.
Righteousness of the officers’ actions can not be judged without tangible
evidence of incidents. Having videos recorded only when an incident occurs will
provide tangible evidence to judge whether officers’ actions were just, without
violating officers’ privacy. One of the difficulties is that it is difficult to
precisely distinguish a moment to start recording a video well enough to
encompass whole situation. Achieving live streaming is optimal but will be
challenging to get done in the given scope of the project. However, once our
proposed function is achieved, it will open many possibilities to extend its
uses and functionalities. When successful, this project will provide optimal
functionality for the use of a wearable body camera in a tactical application. 

\newpage

\bibliographystyle{plain}
\bibliography{references}

\newpage

\appendix
\section{Design Criteria}

\begin{table}[h!]
    \centering
    \caption{Design Criteria}
    \begin{tabular}{lll}
        Design Criteria & Importance & Will/Expect/Stretch\\
        RFID/NFC Holster & High & Will\\
        Video Data Saved Locally & High & Will\\
        RTC synced with GPS & High & Will\\
        Holster/Audio Trigger & High & Will\\
        Video Data Streamed & High & Will\\
        Prevent Tampering & Medium & Stretch\\
        Processor on gun for telemetry & Low & Stretch\\
        Mesh Network for multiple Officers & Low & Stretch\\
    \end{tabular}
\end{table}

\section{Milestones}

\subsection{Milestone 1}
\begin{itemize}
    \item GPS data tested and RTC synced with GPS time
    \item Can show sound level in a room with microphone
    \item RFID is reading proper data on a dedicated power source
    \item Video interface is designed, but may not be functional
    \item Milestone 2:
    \item Video can be taken off of an SD card on the camera
    \item Video is triggered (LED comes on as well as recording starts) with the RFID trigger and sound level
    \item PCB is shipped
    \item Back end for video streaming implemented but not necessarily functional
\end{itemize}

\section{Interface Design}

\begin{lstlisting}
typedef enum day_of_week_t 
{
    SUNDAY,
    MONDAY,
    TUESDAY,
    WEDNESDAY,
    THURSDAY,
    FRIDAY,
    SATURDAY
};


typedef struct rtc_time_t 
{
    uint16_t year;
    uint8_t month;
    day_of_week_t dayOfWeek;
    uint8_t day;
    uint8_t hours;
    uint8_t minutes;
    uint8_t seconds;
};


class RTC_t
{
    public:
    
    void rtcInit();
    rtc_time_t getTime();
    void setTime( rtc_time_t );
    
    private:
    
    rtc_time_t lastTime;
};


typedef enum gps_protocol_t
{
    GPRMC,
    GPGGA,
    GPGSA
};


class GPS_t
{
    public:
    
    void gpsInit();
    void* gpsTask();
    double getLat();
    double getLon();
    rtc_time_t getTime();
    
    private:
    
    gps_protocol_t gpsProtocol;
};


class Wifi_t
{
    public:
    
    void wifiInit();
    int connect();
    int disconnect();
    int send(char* data);
    char* receive();
    
    private:
    int ipAddress;
};


typedef double decibel;


class Microphone_t
{
    public:
    
    void microphoneInit();
    void* microphoneTask();
    decibel getVolume();
    
    private:
};


class Rfid_t
{
    public:
    
    void rfidInit();
    void* rfidTask();
    
    private:
    boolean isTagPresent();
};


class Camera_t
{
    public:
    
    void cameraInit();
    boolean turnCameraOn();
    boolean turnCameraOff();
    boolean isCameraOn();


    private:
};
\end{lstlisting}

\section{Group Agreement}

The team will have weekly meetings where all group members are present on
Tuesdays at 7, a time we have all confirmed via WhenIsGood. We will also have
one midday meeting during the weekend, either Saturdays or Sundays that are
fairly informal where we meet for lunch and discuss high level project goals
and problems. Most meetings will take place in the MESH or 473 lab.. The team
has a WhenIsGood in order to organize meeting times for tasks involving two or
more people, but team members are expected to get work done on their own as
well. The group has its own email at 473.body.cam@umich.edu, as well as a
GroupMe we use to send text message updates to each other. Most of the work
will be done in the 473 lab, but any PCB work will probably be done in the MESH
lab. Any 3D printing will be done in the AERO lab. We also have designated who
will be working on which parts of the project, based on each group member’s
individual strengths. We expect the group members to spend approximately 25-30
hours per week at the beginning of the project, and approximately 40 hours per
week during the second half until the end of the project. The task list is as
follows:

\begin{itemize}
    \item Meeting Minutes - Alex
    \item Group Coordination - Jake
    \item Real Time Clock - Jake
    \item GPS - Alex
    \item RFID Reader - Yun
    \item Microphone - Alex
    \item Camera \begin{itemize}
            \item LED - Alec/Jiqing
            \item Storage - Alec/Jiqing
            \item Low Level Interface - Alec/Jiqing
            \item Streaming - Alex
        \end{itemize}
    \item Mech. CAD and 3D printing - Alex/Alec
    \item PCB Design - Jake/Jiqing
\end{itemize}

\section{Known Conflicts}

\begin{table}[h!]
    \centering
    \caption{Known Conflicts}
    \begin{tabular}{lll}
        Who & What & When\\
        All & Fall Study Break & October 17-20\\
        All & 473 Midterm & October 27\\
        All & Thanksgiving Break & November 25-30\\
        Alex & Out of town & October 25\\
        Alex & Out of town & November 5\\
        Alec & Wisdom Teeth Removal & October 20-22\\
        Yun & Out of town & October 9\\
    \end{tabular}
\end{table}

% [1] 
% 2
% 3
% [2] http://www.govtech.com/data/Oakland-Police-Test-Cloud-Storage-for-Body-Camera-Video.html
% 2 http://www.seattletimes.com/seattle-news/police-tackling-how-to-store-share-body-cam-videos
% 
% 
% [333] Implementing a Body-Worn Camera Program Recommendations and Lessons Learned, Police Executive Research Forum

\end{document}
